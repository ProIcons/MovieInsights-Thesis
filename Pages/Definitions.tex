%Προαιρετικά
\begin{Definitions}
% Ορισμοί εννοιών που μπορεί να είναι χρήσιμοι. Για παράδειγμα:

\begin{description}
    \item [Parallel Programming] Είναι ο παράλληλος προγραμματισμός, και επιτρέπει την ταυτόχρονη εκτέλεση εντολών στον επεξεργαστή. Είναι πού χρήσιμος καθώς ορισμένες ενέργειές σε ένα πρόγραμμα ξοδεύουν παραπάνω χρόνο κάνοντάς άλλες δουλειές κρατώντας τον επεξεργαστή σε αδράνεια. Εκτελώντας παραπάνω εντολές παράλληλα βοηθάει στην πιο αποδοτική χρήση του επεξεργαστή και βελτιώνει δραματικά την απόδοση αυτών των ενεργειών. 
    \item [Thread] είναι ένα νήμα του επεξεργαστή που σου επιτρέπει να τρέχεις κώδικα μέσα του. Όταν τρέχεις διαφορετικά κομμάτια κώδικα σε διαφορετικά threads η εφαρμογή χαρακτηρίζεται Multi threadded, και ο κώδικας τρέχει παράλληλα.
    \item [Thread Safety] Όταν ένας κώδικας τρέχει σε πολλά Threads (νήματα) Ασύγχρονα, και όλα αυτά τα threads έχουν πρόσβαση στα ίδια δεδομένα θα υπάρξει ένα πρόβλημα του συγχρονισμού των δεδομένων. Για παράδειγμα αν έχουμε 2 Threads, το Thread 1 και Thread 2, και υπάρχει στον κώδικα ενας έλεγχος σε μια για παράδειγμα μεταβλητή αν είναι ίση με 0, απο το Thread 1, τον ίδιο χρόνο όμως το Thread 2 αλλάζει την μεταβλητή αυτήν, το Thread 1 δεν θα γνωρίζει για την αλλαγή και μπορεί να προκαλέσει βλάβη με αποτέλεσμα την μη σωστή λειτουργία του κώδικα αυτού. Το Thread Safety είναι "κανόνες" χρησιμοποιώντας τεχνικές για να αποφευχθεί αυτό το φαινόμενο.
    \item [Framework] είναι ένα αφαιρετικό επίπεδο πάνω από ένα ήδη γραμμένο κώδικα το οποίο επιτρέπει στον προγραμματιστή να αλλάξει ρυθμίσεις και να γράψει περαιτέρω κώδικα που να συμπληρώνει τον ήδη υπάρχον κώδικα από το Framework. Η Διαφορά του από την βιβλιοθήκη (Library) είναι οτι η βιβλιοθήκη παρέχει συναρτήσεις και κώδικα για να τις καλέσει ο προγραμματιστής για να διευκολυνθεί, ενώ το Framework καλεί τον κώδικα του προγραμματιστή και τον περιορίζει σε ένα συγκεκριμένο στυλ κώδικα.
    \item [Library] η αλλιώς βιβλιοθήκη, είναι μια συλλογή συναρτήσεων και κλάσεων κώδικα που επιτρέπει στον προγραμματιστή να τις χρησιμοποιήσει για να επιτύχει τους στόχους του με μεγαλύτερη ευκολία
    \item [Browser] είναι ο φυλλομετρητής που επιτρέπει σε εναν χρήστη να πλοηγηθεί σε οποιαδίποτε σελίδα στο διαδύκτιο. Παραδείγματα Browsers είναι: Chrome, FireFox, Edge, Safari κ.λ.π.
    \item [Client] είναι ο αποδέκτης πληροφοριών, που εμφανίζει ένα αποτέλεσμα τον τελικό χρήστη. Ο Client μπορεί να είναι και διαδραστικός που με βάση τις εισαγωγές του χρήστη να παίρνει δεδομένα απο τον Server και να του τα εμφανίζει. Ένα πολύ διάσημο παράδειγμα client θα ήταν ο Browser. Ο Browser στην ουσία επικοινωνεί με έναν διακομιστή (Server) παίρνει ενα αρχείο HTML και το σχεδιάζει με έναν όμορφο τρόπο στο παράθυρο που βλέπει ο χρήστης.
    \item [Server] ή διακομιστής είναι μια υπηρεσία που προσφέρει δεδομένα σε διάφορους clients. Ο Server δεν έχει γραφική διεπαφή, αλλα στέλνει και πέρνει πληροφορίες μέσω προγραμματιστικών διεπαφών (API). 
    \item [API] είναι μια διεπαφή προγραμματιστικού περιβάλλοντος που επιτρέπει διαφορετικές εφαρμογές να επικοινωνούν μεταξύ τους. Από τις πιο διάσημες διεπαφές προγραμματιστικού περιβάλλοντος είναι οι: REST, GraphQL και RPC. Η κάθε μια προσφέρει διαφορετικό τρόπο ανάκτησης και επεξεργασίας δεδομένων.
    \item [MicroService] είναι μια μινιμαλιστική υπηρεσία που ο στόχος της είναι να εκτελεί μια πολύ βασική λειτουργία. Πολλές υπηρεσίες στο διαδίκτυο χρησιμοποιούν αυτό το μοντέλο χρησιμοποιώντας πολλά μικρά MicroServices καθώς αν υπερφορτωθεί το ένα μπορείς να ανοίξεις ένα ακόμα για να μοιράσεις την δουλειά. Έτσι επιτρέπεται με μεγαλύτερη ευκολία το Scaling, 
    \item [Scaling] είναι μια τεχνική για να αυξάνεις το μέγιστο φορτίο που μπορεί να διαχειριστή μια υπηρεσία. Όσο παραπάνω χρήστες χρησιμοποιούν την υπηρεσία τόσο παραπάνω Scaling χρειάζεται.
\end{description}

\end{Definitions}