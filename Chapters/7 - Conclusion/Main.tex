\chapter{Συμπεράσματα}
Αν και η εφαρμογή αυτή είναι πλήρως λειτουργική κρύβει κάποια προβλήματα στην συνολική υλοποίηση της. Έχει έναν αρκετά γρήγορο τρόπο για τον υπολογισμό των δεδομένων που εξαρτάται σημαντικά από την επεξεργαστική ισχύ του μηχανήματος αλλά χρειάζεται να καταναλώσει ένα πάρα πολύ μεγάλο μέρος μνήμης RAM. Επίσης δεν υποστηρίζει ενημέρωση δεδομένων στην παρούσα κατάσταση της. Αν χρειαστεί να ενημερωθούν τα δεδομένα θα πρέπει να υπολογιστούν όλα από την αρχή, και όσα περισσότερα δεδομένα δοθούν για επεξεργασία τόση παραπάνω μνήμη θα καταναλώσει που φτάνει σε ένα σημείο θεωρητικά να μην γίνεται εφικτός ο υπολογισμός αυτών των δεδομένων αν ο όγκος των δεδομένων αυξηθεί σημαντικά.

Η παρούσα υλοποίηση δεν επιτρέπει επιπρόσθετα το Scaling. Αν αυτή η εφαρμογή γινόταν Viral δεν θα υπήρχε εφικτός τρόπος να γίνει Scale και να μπορέσει να εξυπηρετήσει όλους τους χρήστες που θα ζητούσαν δεδομένα.

Πάραυτα τα δεδομένα που προσφέρει είναι πλούσια και ένα άτομο του χώρου της βιομηχανίας του κινηματογράφου μπορεί να τα βρει χρήσιμα για να μπορέσει να επιτελέσει το έργο του με μεγαλύτερη ευκολία.

\section{Στόχοι}
Οι στόχοι της πτυχιακής επιτεύχθηκαν κατά το ήμισυ. Ένας από τους αρχικούς και πιο σημαντικούς στόχους ήταν με έναν συνδυασμό δεδομένων να υπολογιστούν συμπεράσματα για το πως μια ταινία έχει επηρεάσει τον τουρισμό στις χώρες και πόλεις γυρισμάτων. Καθώς θα ήταν ένα πολύ καλό μετρικό για την επίδειξή της δύναμης των OpenData, η ίδια η απουσία των OpenData για αυτά τα δεδομένα, συνετέλεσε στην μη υλοποίησή αυτής της λειτουργίας. Όλοι οι άλλοι στόχοι υλοποιήθηκαν με επιτυχία και υπάρχει ένα πολύ καλό αποτέλεσμα δείχνοντάς ακριβώς γιατί είναι απαραίτητο πολλές εταιρίες να υιοθετήσουν το μοντέλο των OpenData. Όσα περισσότερα δεδομένα κυκλοφορούν ελεύθερα στο διαδίκτυο τόσες παραπάνω ιδέες και εργαλεία θα δημιουργούνται καθημερινά για την αξιοποίηση τους.

\section{Προτάσεις για εξέλιξη}
Όπως είναι φυσιολογικό και προαναφέρθηκε, η εφαρμογή αυτή είναι μεν λειτουργική αλλα δεν είναι σε καμία περίπτωση έτοιμη για ενα Production περιβάλλον. 

\begin{itemize}
    \item Καθώς στην παρούσα υλοποίησή δεν υποστηρίζει Scaling, θα μπορούσε μελλοντικά να το υποστηρίξει χρησιμοποιώντας το μοντέλο των Microservices. 
    \item Θα μπορούσε να αλλάξει ριζικά ο αλγόριθμος υπολογισμού δεδομένων για να επιτρέπει την ενημέρωση των δεδομένων αντί για την συνολική επεξεργασία τους απο την αρχή.
    \item Θα μπορούσε να δημιουργηθεί μια επιπρόσθετη λειτουργία στην υπηρεσία της εισαγωγής δεδομένων, έτσι ώστε να ελέγχει ανα τακτά χρονικά διαστήματα αν υπάρχουν νέα δεδομένα και να τα εισάγει αυτοματοποιημένα.
\end{itemize}

\section{Ο κώδικας}
Ο Κώδικας της εφαρμογής φιλοξενείται στο GitHub στην σελίδα: 

https://github.com/ProIcons/MovieInsights
