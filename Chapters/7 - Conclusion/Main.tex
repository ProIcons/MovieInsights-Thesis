\chapter{Συμπεράσματα}
Αν και η εφαρμογή αυτή είναι πλήρως λειτουργική κρύβει κάποια προβλήματα στην συνολική υλοποίηση της. Έχει έναν αρκετά γρήγορο τρόπο για τον υπολογισμό των δεδομένων που εξαρτάται σημαντικά από την επεξεργαστική ισχύ του μηχανήματος αλλά χρειάζεται να καταναλώσει ένα πάρα πολύ μεγάλο μέρος μνήμης RAM. Επίσης δεν υποστηρίζει ενημέρωση δεδομένων στην παρούσα κατάσταση της. Αν χρειαστεί να ενημερωθούν τα δεδομένα θα πρέπει να υπολογιστούν όλα από την αρχή, και όσα περισσότερα δεδομένα δοθούν για επεξεργασία τόση παραπάνω μνήμη θα καταναλώσει που φτάνει σε ένα σημείο θεωρητικά να μην γίνεται εφικτός ο υπολογισμός αυτών των δεδομένων αν ο όγκος των δεδομένων αυξηθεί σημαντικά.

Η παρούσα υλοποίηση δεν επιτρέπει επιπρόσθετα το Scaling. Αν αυτή η εφαρμογή γινόταν Viral δεν θα υπήρχε εφικτός τρόπος να γίνει Scale και να μπορέσει να εξυπηρετήσει όλους τους χρήστες που θα ζητούσαν δεδομένα.

Πάραυτα προσφέρει μία πληθώρα δεδομένων και ένα άτομο του χώρου της βιομηχανίας του κινηματογράφου μπορεί να τα βρει χρήσιμα για να μπορέσει να επιτελέσει το έργο του με μεγαλύτερη ευκολία.

\section{Στόχοι}
Οι στόχοι της πτυχιακής επιτεύχθηκαν στο σύνολό τους, 
με εξαίρεση κάποιων που επιλέχθηκε να μην υλοποιηθούν λόγω πρακτικών και τεχνικών προβλημάτων. Ένας από τους αρχικούς στόχους ήταν να υπολογιστούν με έναν συνδυασμό δεδομένων, συμπεράσματα για το πως μια ταινία έχει επηρεάσει τον τουρισμό στις χώρες και πόλεις γυρισμάτων. Καθώς θα ήταν ένα πολύ καλό μετρικό για την 
ανάδειξη της δύναμης των OpenData, η ίδια η απουσία των OpenData για αυτά τα δεδομένα, συνετέλεσε στην μη υλοποίησή αυτής της λειτουργίας. Ένας ακόμα αρχικός στόχος που δεν συμπεριλήφθηκε ήταν ένας Live χάρτης ο οποίος θα έδειχνε σε πραγματικό χρόνο για μία ταινία τις χώρες από τις οποίες κατεβάζεται πειρατικά. Αυτός ο στόχος δεν υλοποιήθηκε καθώς για την παρακολούθηση τέτοιων δεδομένων, έπρεπε η εφαρμογή να συμμετέχει σε μια τέτοια παράνομη ενέργεια και προτιμήθηκε η απόρριψη της. Για την περάτωση αυτής της λειτουργίας θα έπρεπε ο Server να υλοποιήσει το πρωτόκολλο Torrent, και να συμμετέχει στον παράνομο διαμοιρασμό των ταινιών έτσι ώστε να μπορεί να αναγνωρίσει μέσω των διευθύνσεων IP από ποιές χώρες κατεβάζονται πειρατικά οι ταινίες. Πολλοί Torrent Tracker καταπολεμούν την "παρακολούθηση" αρνούμενοι να δώσουν Peers σε Clients οι οποίοι δεν ζητάνε δεδομένα. Συνεπώς ο Client θα έπρεπε να κατεβάζει τα αρχεία αλλά και να αποθηκεύει τα δεδομένα καθώς οι Trackers ζητάνε και επαλήθευσή δεδομένων για να βεβαιωθούν ότι οι Clients τα κατεβάζουν. Πέρα από αυτό ο χειρισμός τέτοιων δεδομένων, όπως διευθύνσεις IP, ήθελαν εξαιρετικά ιδιαίτερο χειρισμό ειδικά με την νέα ευρωπαϊκή νομοθεσία GDPR.
Όλοι οι άλλοι στόχοι υλοποιήθηκαν με επιτυχία και υπάρχει ένα πολύ καλό αποτέλεσμα δείχνοντας ακριβώς γιατί είναι απαραίτητο πολλές εταιρίες να υιοθετήσουν το μοντέλο των OpenData. Όσα περισσότερα δεδομένα κυκλοφορούν ελεύθερα στο διαδίκτυο τόσες περισσότερες ιδέες και εργαλεία θα δημιουργούνται καθημερινά για την αξιοποίηση τους.

\section{Προτάσεις για εξέλιξη}
Όπως προαναφέρθηκε, η εφαρμογή αυτή είναι λειτουργική αλλά δεν είναι σε καμία περίπτωση έτοιμη για ένα Production περιβάλλον. Παρακάτω παρατίθενται κάποιες προτάσεις για την εξέλιξη του εν λόγω έργου και μελλοντικών βελτιστοποιήσεων. 

\begin{itemize}
    \item Καθώς η παρούσα υλοποίηση δεν υποστηρίζει Scaling, θα μπορούσε μελλοντικά να το υποστηρίξει χρησιμοποιώντας το μοντέλο των Microservices. 
    \item Θα μπορούσε να αλλάξει ριζικά ο αλγόριθμος υπολογισμού δεδομένων για να επιτρέπει την ενημέρωση των δεδομένων αντί για την συνολική επεξεργασία τους από την αρχή.
    \item Θα μπορούσε να δημιουργηθεί μια επιπρόσθετη λειτουργία στην υπηρεσία της εισαγωγής δεδομένων, έτσι ώστε να ελέγχει ανά τακτά χρονικά διαστήματα αν υπάρχουν νέα δεδομένα και να τα εισάγει αυτοματοποιημένα.
\end{itemize}

\section{Ο κώδικας}
Ο Κώδικας της εφαρμογής φιλοξενείται στο GitHub στην σελίδα: 

https://github.com/ProIcons/MovieInsights
