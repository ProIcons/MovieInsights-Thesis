\subsection{Βήμα 1ο - Δεδομένα από IMDb}
Με τις ρυθμίσεις που δόθηκαν στο σύστημα εισαγωγής δεδομένων για αυτήν την πτυχιακή αρχικά παίρνει ένα συμπιεσμένο αρχείο με δεδομένα βαθμολογιών ταινιών από την υπηρεσία IMDb το αποσυμπιέζει και αποθηκεύει τα δεδομένα στην μνήμη. Η Μορφή του αρχείου είναι TSV που στην ουσία είναι ένα αρχείο σαν πίνακας το οποίο περιέχει στήλες. Οι 3 απαραίτητες στήλες είναι το αναγνωριστικό μιας ταινίας η μιας σειράς (imdbId), η βαθμολογία της ταινίας/σειράς (rating) καθώς και το νούμερο των ψήφων με το οποίο βγήκε αυτή η βαθμολογία (voteCount). Έπειτα παίρνει ένα ακόμα συμπιεσμένο αρχείο από την υπηρεσία IMDb του οποίου η μορφή είναι επίσης TSV με στήλες το αναγνωριστικό μιας ταινίας η μιας σειράς (imdbId) και ο τύπος (titleType). Ο τύπος διευκρινίζει εάν αν είναι ταινία, σειρά ή κάτι άλλο. 

Αφού φορτώσει και τα 2 αρχεία στη μνήμη ξεχωρίζει τις εγγραφές που αναφέρουν μόνο ταινίες και μετά ξεχωρίζει τις εγγραφές οι οποίες έχουν αριθμό ψήφων μεγαλύτερο ή ίσο με μια σταθερά MINIMUM\_VOTES\_THRESHOLD και κρατάει τα 3 αρχικά πεδία του πρώτου αρχείου. Η σταθερά αυτή έχει καθοριστεί ως 1000 ψήφοι.


\begin{figure}[h]
  \centering
  \includegraphics[width=150mm]{Chapters/5 - Architecture/Import/Images/imdb_flowchart.png}
  \caption{Διάγραμμα ροής δεδομένων από IMDB}
  \label{flowchart:imdbImport}
\end{figure}