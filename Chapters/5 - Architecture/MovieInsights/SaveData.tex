\subsection{Βήμα 3 - Αποθήκευση δεδομένων}
Έχοντάς συλλέξει, κατηγοριοποιήσει και υπολογίσει όλα τα απαραίτητα δεδομένα, τα δεδομένα αυτά με ελάχιστες μετατροπές στέλνονται στον Hibernate για να αποθηκευτούν στην βάση δεδομένων έτσι ώστε να τεθεί η εφαρμογή σε κατάσταση ετοιμότητας. 

Για την αποθήκευση δεδομένων δημιουργείται μια νέα "συναλλαγή". Η συναλλαγή αυτή βεβαιώνει ότι αν κάποιο από τα δεδομένα αυτά δεν μπορούσε για οποιονδήποτε λόγω να αποθηκευτεί στην βάση δεδομένων, να ακυρωθεί όλη αυτή η διαδικασία και να τερματίσει η εφαρμογή με το ανάλογο μήνυμα σφάλματος, έτσι ώστε να μπορεί να επιλυθεί απο κάποιον διαχειριστή. Η αποθήκευση μερικών δεδομένων στην βάση δεδομένων θα είχε καταστροφικές συνέπειες στην λειτουργία της εφαρμογής. 

Μετά το πέρας της αποθήκευσης, γίνεται εκκαθάριση του Cache και ζητείται να γίνει και εκκαθάριση μνήμης από τον Garbage Collector το JVM, ώστε να ελευθερωθεί μνήμη για την καλύτερη λειτουργία της εφαρμογής. Πολλές φορές αυτό δεν είναι δυνατό ανάλογα με το JVM που χρησιμοποιείται και έτσι μετά την αρχικοποίηση στην περίπτωση που δεν υπάρχει διαθέσιμη μνήμη συστήνεται η επανεκκίνηση της εφαρμογής.

