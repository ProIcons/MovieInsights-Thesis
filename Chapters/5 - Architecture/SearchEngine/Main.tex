\section{Μηχανή Αναζήτησης}
Η μηχανή αναζήτησης είναι μια σημαντική υπηρεσία για την εφαρμογή και επιτρέπει στον χρήστη την αναζήτηση και αυτόματη συμπλήρωση της αναζήτησης ατόμων, εταιριών παραγωγής, χωρών παραγωγής και ειδών ταινιών. 

Η μηχανή αναζήτησης έχει την δυνατότητα να συνδεθεί σε μια εξωτερική βαση δεδομένων για την ανάκτηση δεδομένων είτε να χρησιμοποιήσει την εσωτερική της βάση δεδομένων. Για την βελτιστοποίηση της εμπειρίας του χρήστη η μηχανή αναζήτησης χρησιμοποιεί την δικιά της βάση δεδομένων, καθώς είναι πολύ πιο γρήγορη απο μια συμβατική SQL βάση δεδομένων, με το μειονέκτημα της διπλοτυπίας δεδομένων. 

Για περαιτέρω βελτιστοποίηση δημιουργήθηκαν ξεχωριστά Indices για την μηχανή αναζήτησης και ξεχωριστά Entities για την βάση δεδομένων. Έχοντάς ως κύρια πηγή της αλήθειας τα Entities της βάσης δεδομένων SQL, τα Indicies της μηχανής αναζήτησης έχουν τα λιγότερα δυνατά δεδομένα που χρειάζονται για την αναζήτηση, μειώνοντας σημαντικά τον απαιτούμενο χώρο αποθήκευσης.

Καθώς όμως τα δεδομένα δεν βρίσκονται σε μια μόνο βάση δεδομένων, χρειάζεται ο συγχρονισμός τους καθ όλη την λειτουργία της εφαρμογής. Αυτό γίνεται εύκολα καθώς τα δεδομένα που χρησιμοποιούνται στην αναζήτηση, αλλάζουν μόνο όταν τρέχει το σύστημα εισαγωγής δεδομένων και αυτό γίνεται μόνο στην εκκίνηση της εφαρμογής.

Πάραυτα καθώς η μηχανή αναζήτησης είναι ένα εξωτερικό σύστημα, όχι άμεσα επιβλεπόμενο απο την εφαρμογή, τα δεδομένα της βάσης δεδομένων της, μπορούν να αλλάξουν και να αλλοιωθούν ανά πάσα στιγμή. Για αυτόν τον λόγο έχει δημιουργηθεί μια υπηρεσία που οι αποκλειστικές αρμοδιότητες της είναι να συγχρονίζει τα δεδομένα της κύριας βάσης δεδομένων με την βάση δεδομένων της μηχανής αναζήτησης. Αυτή η υπηρεσία ενεργεί μια φορά στην εκκίνηση ακριβώς πριν τεθεί η εφαρμογή σε κατάσταση ετοιμότητας, και ενεργεί όταν ένας διαχειριστής το ζητήσει. 

Η Μηχανή Αναζήτησης πέρα απο τον συγχρονισμό δεδομένων επιτρέπει τον χρήστη να αναζητήσει κάτι, αλλα παράλληλα επιτρέπει και την αυτόματη συμπλήρωση. Για να λειτουργήσει η Αυτόματη συμπλήρωση όλα τα Indices έπρεπέ να ρυθμιστούν ανάλογα.

\begin{figure}[H]
    \begin{jsoncode}
{"analysis": {
  "filter": {
    "autocomplete_filter": {
      "type": "edge_ngram",
      "min_gram": 1,
      "max_gram": 40
    }
  },
  "analyzer": {
    "autocomplete_search": {
      "type": "custom",
      "tokenizer": "standard",
      "filter": ["lowercase"]
    },
    "autocomplete_index": {
      "type": "custom",
      "tokenizer": "standard",
      "filter": ["lowercase", "autocomplete_filter"]
    }
  }
}}
    \end{jsoncode}
    \caption{Ρυθμίσεις ενός Index της ElasticSearch}
   \label{config:es}
\end{figure}

Για να λειτουργήσει η αυτόματη συμπλήρωση κάθε λέξη και πρόταση μέσα σε αυτά τα Indices έπρεπε να χωριστεί σε ngrams. Όπως φαίνεται στο σχήμα \ref{config:es} στην σειρά 18, δηλώθηκε ενα φίλτρο το οποίο σπάει τις λέξεις και προτάσεις σε ngrams και ετσι επιτρέπει την ρύθμιση ενός πεδίου ενός Index για να μπορεί να αναζητηθεί με αυτόματη συμπλήρωση.

Το επόμενο βήμα ήταν να δηλωθούν τα πεδία στα Indices που θα έχουν τα φίλτρα του autocomplete, όπως φαίνεται στο σχήμα \ref{code:index}

\begin{figure}[h]
    \begin{javacode}
@MultiField(
    mainField = @Field(
        type = Text, 
        fielddata = true,
        analyzer = "autocomplete_index", 
        searchAnalyzer = "autocomplete_search"),
    otherFields = {@InnerField(suffix = "verbatim", type = Keyword)})
private String name;
    \end{javacode}
    \caption{Annotations ενός Index της ElasticSearch}
   \label{code:index}
\end{figure}

Με αυτόν τον τρόπο δηλώνονται τα Indices στον κώδικα με τις σωστές ρυθμίσεις, και η βιβλιοθήκη που χρησιμοποιήθηκε για να επικοινωνήσει με την ElasticSearch, Jest αρχικοποιεί τα Indices στην ElasticSearch και αφου προστεθούν δεδομένα απο τον συγχρονισμό των 2 βάσεων, η μηχανή αναζήτησης είναι έτοιμη για χρήση.

Καθώς όπως προαναφέρθηκε η ElasticSearch είναι ένα αυτόνομο σύστημα πρέπει να του δοθεί ενα ανάλογο Query που να μπορεί να το καταλάβει για να μπορέσει να σερβίρει τα δεδομένα. Αυτό το κάνει η υπηρεσία SearchService.

Όλα τα entities της εφαρμογής έχουν το ανάλογο Service για να επιτρέψουν την ανάκτηση, μορφοποιήση, δημιουργεία αλλα και διαγραφή των δεδομένων τους, γνωστό ως CRUD. Τα Entities που έχουν ανάλογα Indicies μοιράζονται τα ίδια Services για λόγους απλότητας και συγκέντρωσης των αρμοδιοτήτων σε ενα σημείο. 

Κάθε υπηρεσία ενός Entity που έχει παράλληλα και Index, κάνει εγγραφή στο SearchService κατα την εκκίνηση, έτσι ώστε οταν φτάσει η στιγμή που θα γίνει ένα ερώτημα αναζήτησης, το SearchService να γνωρίζει απο ποια Indices θα παρθούν τα δεδομένα. Το κάθε Service που γίνεται register ορίζει επίσης ένα Query. Με αυτόν τον τρόπο η μηχανή αναζήτησης δεν χρειάζεται να γνωρίζει τίποτα για τα Indices, απλά παίρνει τα ερωτήματα απο όλα τα Services που δηλώνουν Indices, τα συνδιάζει και τα στέλνει στην ElasticSearch. Οταν πάρει τα δεδομένα, τα αλλάζει μορφή και τα στέλνει στον Client σε μια διαφορετική μορφή για την βελτιστοποίηση του μεγέθους της απάντησης. 

Όταν γίνεται η αυτόματη συμπλήρωση, για κάθε γράμμα που πληκτρολογείται στον Client, γίνεται και ένα ερώτημα στον Server και στην Μηχανή Αναζήτησης. Είναι πολύ σημαντικό και η μηχανή αναζήτησης να σερβίρει τα δεδομένα πολύ γρήγορα, αλλα και ο Server να απαντάει πολύ γρήγορα. Πέρα απο την ταχύτητα που θα απαντάει ο Server, πρέπει επίσης η απάντηση να είναι μικρή έτσι ώστε πρώτον να εξοικονομεί δεδομένα απο clients οι οποίοι βρίσκονται σε Metered Connections, και να στέλνονται πιο γρήγορα τα αποτελέσματα σε Clients οι οποίοι έχουν χαμηλή ταχύτητα Internet.

Για παράδειγμα το αποτέλεσμα της αναζήτησης που επιστρέφει η μηχανή αναζήτησης όπως φαίνεται στο σχήμα \ref{result:es} έχει μέγεθος 420 bytes για ένα αποτέλεσμα με μέγεθος αποτελέσματος 232 bytes, σε αντίθεση με το αποτέλεσμα του SearchService όπως φαίνεται στο σχήμα \ref{result:ss} με μέγεθος 166 bytes, για ένα αποτέλεσμα με μέγεθος αποτελέσματος 123 bytes. Είναι εμφανές λοιπόν οτι με την βελτιστοποίηση το μέγεθος της απάντησης είναι πολύ μικρότερο.

\begin{figure}[H]
    \begin{jsoncode}
{
    "took": 23,
    "timed_out": false,
    "_shards": {
        "total": 1,
        "successful": 1,
        "skipped": 0,
        "failed": 0
    },
    "hits": {
        "total": 37470,
        "max_score": 846.1094,
        "hits": [{
            "_index": "person",
            "_type": "person",
            "_id": "323",
            "_score": 846.1094,
            "_source": {
                "popularity": 35.808,
                "name": "Some Person",
                "profilePath": "/cckcYc2v0yh1tc9QjRelptcOBko.jpg",
                "id": 323
            }
        }]
    }
}
    \end{jsoncode}
    \caption{Αποτέλεσμα αναζήτησης ElasticSearch}
   \label{result:es}
\end{figure}
\begin{figure}[H]
    \begin{jsoncode}
{
    "_": [{
        "e": [{
            "id": 323,
            "name": "Some Person",
            "popularity": 35.808,
            "profilePath": "/cckcYc2v0yh1tc9QjRelptcOBko.jpg"
        }],
        "i": "PERSON"
    }]
}
    \end{jsoncode}
    \caption{Αποτέλεσμα Αναζήτησης SearchService}
   \label{result:ss}
\end{figure}
