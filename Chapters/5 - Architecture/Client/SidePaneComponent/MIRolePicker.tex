Το MIRolePicker Component εμφανίζεται μόνο όταν η επιλεγμένη κατηγορία είναι "ανά άτομο". Αποτελείται από 4 κουμπιά μονής επιλογής. Αυτό σημαίνει ότι όταν πατηθεί ένα κουμπί επιλέγεται μένει πατημένο και οποιαδήποτε άλλο κουμπί ήταν πατημένο πρωτύτερα αφαιρείται η επιλογή του. Λειτουργεί ακριβώς με τον ίδιο τρόπο με τα παραδοσιακά Radio Buttons. Τα 4 κουμπιά αντιστοιχούν στους 4 ρόλους που μπορεί να έχει ένα άτομο όπως Ηθοποιός, Σκηνοθέτης, Συγγραφέας και Παραγωγός όπως φαίνεται στο σχήμα \ref{layout:mirolepicker}. Όταν ένα Role είναι επιλεγμένο το ανάλογο κουμπί επιλέγεται και εμφανίζεται με χρώμα μπλε, όταν δεν υπάρχει το συγκεκριμένο Role στο άτομο είναι απενεργοποιημένο και εμφανίζεται με χρώμα ανοιχτό γκρι, και αν υπάρχει και δεν είναι επιλεγμένο εμφανίζεται με χρώμα σκούρο γκρι και ο κώδικας επιλογής φαίνεται στο σχήμα. Όταν αλλάζει η επιλογή ο κώδικας που αλλάζει την διεύθυνση φαίνεται στο σχήμα \ref{code:mirolepicker_urlchanger}.
\begin{figure}[h]
  \centering
  \includegraphics[width=55mm]{Chapters/5 - Architecture/Client/Images/mirolepicker.png}
  \caption{MIRolePicker Component}
  \label{layout:mirolepicker}
\end{figure}

\begin{figure}[H]
    \begin{TypeScriptcode}
private onCreditSelect = (credit: CreditRole) => {
  const activeView = this.props.rootState.dashboardState.activeView() as MovieInsightsPerPersonState;
  this.props.history.push(AppUtils.|$\textbf{generateNavigationLink}$|(activeView._activeEntity.person, credit, activeView.isPerYear ? activeView.activeYearEntity.entity : null));
}
    \end{TypeScriptcode}
    \caption{Αλγόριθμος αλλαγής διεύθυνσης από το MIRolePicker Component.}
   \label{code:mirolepicker_urlchanger}
\end{figure}