\chapter{OpenData}

Open Data είναι η ιδέα ότι ορισμένα δεδομένα πρέπει να είναι διαθέσιμα δωρεάν σε όλον τον κόσμο για να τα χρησιμοποιεί ή να τα αναπαράγει όπως επιθυμεί χωρίς περιορισμούς από νόμους πνευματικών δικαιωμάτων, πατέντες ή μηχανισμούς ελέγχου. Οι στόχοι του κινήματος των open-source data, είναι κοινοί με των άλλων κινημάτων open-source όπως open-source software, hardware κ.ο.κ. \citep{wiki:opendata}.
Παραδόξως η ανάπτυξη του κινήματος open-source data είναι παράλληλη με την αύξηση των δικαιωμάτων πνευματικής ιδιοκτησίας.

Αν κάποιος αναρωτιέται γιατί είναι τόσο σημαντικό να είναι σαφές τι σημαίνει Ανοιχτά Δεδομένα και σε τι είναι χρήσιμος αυτός ο ορισμός, υπάρχει μια απλή απάντηση: η διαλειτουργικότητα. Η διαλειτουργικότητα δηλώνει τη δυνατότητα διαφορετικών συστημάτων να λειτουργούν μαζί (διαλειτουργούν). Σε αυτή τη συγκεκριμένη περίπτωση, γίνεται αναφορά στη δυνατότητα να διαλειτουργούν –ή να αναμιγνύουν- διαφορετικά σύνολα δεδομένων.

Η διαλειτουργικότητα (interoperability) είναι σημαντική επειδή επιτρέπει στις διαφορετικές συνιστώσες να λειτουργούν μαζί. Αυτή η δυνατότητα διαμοίρασης και σύνδεσης συνιστωσών έχει θεμελιώδη σημασία για τη δόμηση μεγαλύτερων και πιο πολύπλοκων συστημάτων. Χωρίς τη δυνατότητα διαλειτουργικότητας αυτό γίνεται σχεδόν αδύνατο; απόδειξη αποτελεί η διάσημη ιστορία του Πύργου της Βαβέλ, όπου η αδυναμία επικοινωνίας (διαλειτουργίας) οδήγησε στην ολοκληρωτική κατάρρευση της προσπάθειας οικοδόμησης του. \citep{github:opendata}

Δυστυχώς, το τοπίο των OpenData είναι πολύ θολό καθώς οι εταιρίες που προσφέρουν αυτά τα δεδομένα, πέρα από τους κυβερνητικούς οργανισμούς δεν ακολουθούν κάποια στάνταρ, και υπάρχουν διαφορετικοί περιορισμοί στην χρήση και αναπαραγωγή αυτών των δεδομένων. 

Είναι σημαντικό να υπάρξει ένα πρότυπο ή μια αρμόδια αρχή που να υπαγορεύει ακριβώς πως θα χρησιμοποιούνται τα OpenData  γενικότερα αλλά επίσης είναι σημαντικό όλο και περισσότερες εταιρίες να υιοθετήσουν αυτό το μοντέλο, καθώς με το "άνοιγμα" των δεδομένων δημιουργούνται νέες ιδέες για νέες τεχνολογίες και αναπτύσσεται γενικότερα η τεχνολογία και η κοινωνία.

Τα open source data (OpenData) δεν είναι πολύ διαφορετικά από το open source sofwtare (ή όπως χρησιμοποιείται ευρέως) OpenSource. Το κίνημα του open source software είναι ένα κίνημα για δωρεάν και ελεύθερο λογισμικό που η κοινωνία μπορεί να συμμετέχει στην εξέλιξη του και στην διασφάλιση της ποιότητας του. Το κίνημα των open source data είναι ένα κίνημα για δωρεάν και ελεύθερα δεδομένα τα οποία μπορεί να αξιοποιήσει η κοινωνία για την δημιουργία open source software. 

Έχοντας παραπάνω πηγές open source data, θα δημιουργούνται περισσότερες ιδέες και παραπάνω εργαλεία που θα συμβάλλουν σημαντικά στην εξέλιξη μας. Όσες περισσότερες ιδέες και εργαλεία τόσα παραπάνω open source software θα γεννηθούν για την αξιοποίηση αυτών των δεδομένων. Η παρούσα πτυχιακή είναι ένα project το οποίο γεννήθηκε μέσα από τη δυνατότητα αξιοποίησης των open source data. Δεν θα ήταν εφικτή η υλοποίησή της χωρίς αυτά. 