\section{Stories Γενικού Χρήστη}
\definecolor{gbg}{RGB}{71,153,231}
\begin{tcolorbox}[colback=fg!10!white,colframe=gbg!100!white,
	title=1η ιστορία]
Σαν Χρήστης θα ήθελα να μου προσφέρονται αναλυτικά στοιχεία ανά 4 κατηγορίες: ανά Συντελεστή, ανά Χώρα Παραγωγής, ανά Εταιρία Παραγωγής και ανά Είδος Ταινίας έτσι ώστε να έχω μια πιο μεγάλη ευελιξία στις αναζητήσεις μου και στην προβολή των στοιχείων αυτών.
\end{tcolorbox}

\begin{tcolorbox}[colback=fg!10!white,colframe=gbg!100!white,
	title=2η ιστορία]
Σαν Χρήστης θα ήθελα να μου προσφέρονται αυτά τα αναλυτικά στοιχεία στις 4 κατηγορίες και ανά χρόνο για να ειδικεύσω περισσότερο την αναζήτηση μου έτσι ώστε να έχω μια πιο ολοκληρωμένη εικόνα των δεδομένων που ψάχνω.
\end{tcolorbox}

\begin{tcolorbox}[colback=fg!10!white,colframe=gbg!100!white,
	title=3η ιστορία]
Σαν Χρήστης θα ήθελα να υπάρχει ένα πεδίο αναζήτησης για να μπορώ να ψάχνω συντελεστές , χώρες παραγωγής, εταιρίες παραγωγής και είδη ταινιών έτσι ώστε να με διευκολύνει στην εύρεση των ζητούμενων στοιχείων.
\end{tcolorbox}

\begin{tcolorbox}[colback=fg!10!white,colframe=gbg!100!white,
	title=4η ιστορία]
Σαν Χρήστης θα ήθελα όταν επιλέξω ένα από τα εμφανιζόμενα στοιχεία σε μια κατηγορία, να ανανεωθούν όλα τα δεδομένα βάση του επιλεχθέντος στοιχείου, ώστε να έχω μια στοχευμένη ανασκόπηση των δεδομένων σύμφωνα με την επιλογή μου.
\end{tcolorbox}

\begin{tcolorbox}[colback=fg!10!white,colframe=gbg!100!white,
 	title=5η ιστορία]
Καθώς υπάρχουν ήδη εργαλεία τα οποία προσφέρουν αναλυτικά στοιχεία για ταινίες, ως χρήστης δε θα ήθελα να δω έναν κλώνο αυτών των υπηρεσιών. Θα ήθελα, όταν θα πατήσω σε μία από αυτές τις ταινίες, αφού μου εμφανιστεί ένα παράθυρο με τα πολύ βασικά στοιχεία, να μου προσφερθεί η επιλογή για ανακατεύθυνση σε μια από τις υπάρχουσες υπηρεσίες, σε περίπτωση που θελήσω περισσότερες πληροφορίες.
\end{tcolorbox}


\begin{tcolorbox}[colback=fg!10!white,colframe=gbg!100!white,
 	title=6η ιστορία]
Σαν χρήστης θα ήθελα όταν θα μπω στην ιστοσελίδα, να υπάρχει ένας παγκόσμιος χάρτης στον οποίο να αναδεικνύεται ποιά χώρα έχει τις περισσότερες ταινίες. Επίσης, όταν μετακινήσω τον κένσορα του ποντικιού πάνω από μια χώρα θα ήθελα να εμφανιστεί ένα μικρο παραθυράκι (tooltip) πάνω από την χώρα, το οποίο να αναγράφει σε πόσες ταινίες έχει συμμετάσχει στην παραγωγή τους αυτή η χώρα. Αν πατήσω στην χώρα θα ήθελα να εμφανιστούν όλα τα συνδυαστικά στοιχεία για αυτήν την χώρα. Οι λειτουργίες αυτές στοχεύουν στην άμεση ανάδειξη των στατιστικών χρησιμοποιώντας γεωγραφικά κριτήρια. Η ύπαρξη του παγκόσμιου χάρτη έχει ως σκοπό τη διευκόλυνση της αναζήτησης μέσω της οπτικοποίησης των δεδομένων.
\end{tcolorbox}
