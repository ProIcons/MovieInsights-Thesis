\chapter{Απαιτήσεις εφαρμογής}
Με όλη αυτή την ανάπτυξη του κινηματογράφου και των τεχνολογιών που δημιουργήθηκαν γύρο από αυτόν, δημιουργήθηκαν εργαλεία που μας προσφέρουν αναλυτικά δεδομένα για ταινίες, σειρές κ.ο.κ. Αυτά τα εργαλεία έχουν πλούσιες βάσεις δεδομένων με πολλές εγγεγραμμένες ταινίες αλλά προσφέρουν στοιχεία μόνο για τις ταινίες αυτές. Δεν προσφέρουν συνδυαστικά και συγκεντρωτικά στοιχεία για όλες τις οντότητες που περιβάλλουν τον κινηματογράφο. Οντότητες όπως ηθοποιοί, εταιρίες παραγωγής, χώρες παραγωγής και είδη ταινιών. Υπάρχουν τόσα αναξιοποίητα δεδομένα που με τον συνδυασμό τους θα μπορούσε να δημιουργηθεί μια υπηρεσία η οποία θα μπορούσε να προσφέρει αναλυτικά στοιχεία σε βάθος χρόνου όλων αυτών των οντοτήτων. Έτσι λοιπόν γεννήθηκε η ιδέα της εφαρμογής MovieInsights. 

Με την παρούσα εφαρμογή θέλω να δημιουργήσω ένα εργαλείο το οποίο θα μπορεί να χρησιμοποιηθεί απο τους ειδικούς του χώρου της βιομηχανίας του κινηματογράφου αλλά παράλληλα και να δείξω πως με την αξιοποίηση αυτών των δεδομένων μπορούν να δημιουργηθούν ακόμη περισσότερες τεχνολογίες για την διευκόλυνσή των εργασιών στο χώρο. Επίσης δεν θα ήθελα να δημιουργήσω μια τεχνολογία που θα μπορούσαν μόνο οι ειδικοί να χρησιμοποιήσουν αλλά να μπορεί και ο απλός ο κόσμος να έχει πρόσβαση και να μπορεί να καταλάβει αυτόν τον μεγάλο όγκο δεδομένων σχεδιάζοντας μια φιλική προς τον χρήστη διεπαφή.


Σαν χρήστης θα ήθελα να μπορώ να μπαίνω σε μια ιστοσελίδα και να βλέπω αναλυτικά τα στοιχεία τα οποία να περιέχουν τον πιο δημοφιλή και λιγότερο δημοφιλή ηθοποιό, παραγωγό, σκηνοθέτη και συγγραφέα, θα ήθελα επίσης να μπορώ να δω το πιο δημοφιλές και λιγότερο δημοφιλές είδος ταινίας, την πιο δημοφιλή και λιγότερη δημοφιλή χώρα και εταιρία παραγωγής, να δω τις ταινίες με τα περισσότερα και λιγότερα έσοδα και έξοδα και τις ταινίες με την μεγαλύτερη και χαμηλότερη βαθμολογία. Επειδή υπάρχουν ήδη εργαλεία τα οποία προσφέρουν αναλυτικά στοιχεία για ταινίες, δεν θα ήθελα να δω εναν κλώνο αυτών των υπηρεσιών αλλά όταν θα πατήσω σε μια από αυτές τις ταινίες να μου εμφανιστεί ένα παράθυρο με τα πολύ βασικά στοιχεία, και αν θέλω περισσότερα να μου προσφερθεί η επιλογή για ανακατεύθυνση σε μια από τις υπάρχουσες υπηρεσίες.

Όλα αυτά τα θέλω και συνδυαστικά. Για παράδειγμα θέλω να δω όλα τα παραπάνω στοιχεία με σημείο αναφοράς έναν ηθοποιό, δηλαδή με ποίους άλλους ηθοποιούς έχει συνεργαστεί αυτός ο ηθοποιός η με ποιους άλλους σκηνοθέτες, παραγωγούς και συγγραφείς. Το ίδιο και για τους σκηνοθέτες, παραγωγούς και συγγραφείς. Και όχι μόνο στους συντελεστές. Να μπορώ να βλέπω στοιχεία για μια χώρα. Με ποιους συντελεστές έχει συνεργαστεί παραπάνω αυτή η χώρα, με ποια άλλη χώρα έχει κάνει τις περισσότερες συμπαραγωγές η ποίο είναι το δημοφιλέστερο είδος ταινίας για αυτήν την χώρα. 

Γενικά θα ήθελα όλα αυτά τα στοιχεία συνδυαστικά με όλα αυτά τα στοιχεία. Επίσης θα ήθελα να προσφέρονται αυτά τα συνδυαστικά στοιχεία και ανά έτος.

Όταν θα μπώ στην ιστοσελίδα θα ήθελα να υπάρχει ένας παγκόσμιος χάρτης που από τον οποίο να μπορώ να καταλάβω ποια χώρα έχει τις περισσότερες ταινίες. Επίσης όταν μετακινήσω τον κένσορα του ποντικιού πάνω από μια χώρα θα ήθελα να δω να εμφανιστεί ένα μικρο παραθυράκι (tooltip) πάνω απο την χώρα και να μου αναγράφει σε πόσες ταινίες έχει συμμετάσχει στην παραγωγή τους αυτή η χώρα. Αν πατήσω στην χώρα να εμφανιστούν όλα τα συνδυαστικά στοιχεία για αυτήν την χώρα. 

Θα ήθελα επίσης να μπορώ να δω ανα κατηγορία τα μέσα και συνολικά, έσοδα και έξοδα ανα χρόνο αλλα και γενικά, να δω το μέσο και συνολικό καθαρό κέρδος. Επίσης να μπορώ να δω την μέση βαθμολογία των ταινιών ανα χρόνο αλλα και συνολικά, και την τάση αύξησης η μείωσης της βαθμολογίας (Mean) ανά χρόνο.

Επίσης στην κατηγορία των συντελεστών θα ήθελα να υπάρχουν στοιχεία για αυτόν τον συντελεστή σε οποιοδήποτε ρόλο έχει συμμετέχει στην καριέρα του, και κάθε φορά να μπορώ να επιλέξω για ποιόν ρόλο να δω τα στοιχεία αυτά.

Από τα εμφανιζόμενα δεδομένα των πιο δημοφιλών και λιγότερο δημοφιλών οντοτήτων όταν πατάω πάνω τους θα ήθελα να ανακατευθύνομαι στα συνδυαστικά στοιχεία αυτών. 

Επίσης θα ήθελα να υπάρχει ένα πεδίο αναζήτησης για να μπορώ να ψάχνω συντελεστές , χώρες και εταιρίες.
