\subsection{Redux}

\begin{figure}[h]
  \centering
  \includegraphics[width=75mm]{Chapters/3 - Technologies/Images/redux-logo.png}
  \caption{Λογότυπο Redux}
  \label{fig:redux-logo}
\end{figure}

Το Redux είναι ένα προβλέψιμο state container για εφαρμογές JavaScript. Θεωρείται προβλέψιμο διότι το State (κατάσταση της εφαρμογής) είναι αμετάβλητο και σε κάθε πρόθεση αλλαγής γίνεται αντικατάσταση. Με αυτόν τον τρόπο καθίσταται δυνατή η αλλαγή του State της εφαρμογής οπουδήποτε μέσα στο ιστορικό (undo,redo). Επίσης διευκολύνει το debugging καθώς δεν υπάρχει mutability και επομένως η χρονική στιγμή και οι παράγοντες κάθε αλλαγής είναι ορατοί ανά πάσα στιγμή/συνεχώς.

Το Redux δεν σχεδιάστηκε για την React, αλλά μεγάλο μέγεθος εφαρμογών γραμμένων σε React το χρησιμοποιούν λόγω
των μεγάλων δυνατοτήτων και της απλότητας του. Οι δημιουργοί του Redux δημιούργησαν επίσης ένα επίσημο πακέτο για την εύκολη επικοινωνία της React με το Redux, με όνομα react-redux. 

Το react-redux χρησιμοποιήθηκε στο Frontend της Πτυχιακής,
μαζί με κάποιες ακόμα επεκτάσεις του για την λήψη των δεδομένων από το Backend καθώς και για το Caching. Χρησιμοποιήθηκε επίσης και για την δρομολόγηση των αποτελεσμάτων που θα εμφανίζονται.