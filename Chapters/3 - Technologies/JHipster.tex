\section{JHipster}

\begin{figure}[h]
  \centering
  \includegraphics[]{Chapters/3 - Technologies/Images/jhipster-logo.png}
  \caption{Λογότυπο JHipster}
  \label{fig:jhipster-logo}
\end{figure}

Το JHipster είναι ένα εργαλείο αυτόματης παραγωγής κώδικα. Στόχος του είναι η αυτόματη δημιουργία αρχείων κώδικα τα οποία επαναλαμβάνονται για κάθε εφαρμογή του είδους.

Για παράδειγμα, όταν θέλουμε να δημιουργήσουμε μια υπηρεσία API στην Java, χρησιμοποιώντας μια βάση δεδομένων SQL, κατά προτίμηση
% (?)
πρέπει αρχικά να επιλέξουμε ένα Web Framework, για να διευκολύνει την διαχείριση και την ανάπτυξη του προγράμματος. Αφού επιλέξουμε αυτό το Framework, πρέπει να το ρυθμίσουμε. Στην συγκεκριμένη περίπτωση πρέπει να το ρυθμίσουμε να στέλνει δεδομένα τύπου JSON ή XML αφού αναφερόμαστε σε μια υπηρεσία API. Πρέπει επίσης να ρυθμίσουμε την σύνδεση με την βάση δεδομένων, πρέπει να ρυθμίσουμε πως θα εμφανίζονται οι ημερομηνίες στις απαντήσεις... κ.ο.κ. 

Το Jhipster δημιουργεί όλα αυτά τα αρχεία και τις ρυθμίσεις δίνοντας την επιλογή περαιτέρω ρύθμισης και με αυτόν τον τρόπο εξοικονομεί
χρόνο από την ρύθμιση της υποδομής και επιτρέπει άμεση εστίαση
στην υλοποίηση της εφαρμογής. 

Εκτός από τις ρυθμίσεις επιτρέπει τη δήλωση του Μοντέλου
της εφαρμογής σου στην JDL DSL του, και μπορεί να δημιουργήσει τα ανάλογα Entities, Services Και Controllers και στο Frontend και στο Backend.

Η πτυχιακή αυτή αρχικά δημιουργήθηκε με το JHipster χρησιμοποιώντας το Spring Framework για το backend και την React Library για το Frontend. Επιπρόσθετα προσέθεσε ρυθμίσεις και για αρκετά άλλες τεχνολογίες που χρησιμοποιήθηκαν.

