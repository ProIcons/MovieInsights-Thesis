\section{JHipster}

\begin{figure}[h]
  \centering
  \includegraphics[]{Chapters/3 - Technologies/Images/jhipster-logo.png}
  \caption{Λογότυπο JHipster}
  \label{fig:jhipster-logo}
\end{figure}

Το JHipster είναι ένα εργαλείο αυτόματης παραγωγής κώδικα. Στόχος του είναι η αυτόματη δημιουργία αρχείων κώδικα τα οποία επαναλαμβάνονται για κάθε εφαρμογή του είδους.

Για παράδειγμα, για την δημιουργία μιας υπηρεσίας API στην Java, χρησιμοποιώντας μια βάση δεδομένων SQL, κατά προτίμηση πρέπει αρχικά να επιλεγεί ένα Web Framework, για να διευκολύνει την διαχείριση και την ανάπτυξη του προγράμματος. Αφού επιλεγεί αυτό το Framework, πρέπει να ρυθμιστεί ανάλογα για να καλύψει τις ανάγκες της εφαρμογής. Στην συγκεκριμένη περίπτωση πρέπει να ρυθμιστεί να στέλνει δεδομένα τύπου JSON ή XML αφού ο στόχος είναι η δημιουργία μιας υπηρεσίας API. Πρέπει επίσης να ρυθμιστεί η σύνδεση με την βάση δεδομένων, πρέπει να ρυθμιστεί πως θα εμφανίζονται οι ημερομηνίες στις απαντήσεις... κ.ο.κ. 

Το JHipster δημιουργεί όλα αυτά τα αρχεία και τις ρυθμίσεις δίνοντας την επιλογή περαιτέρω ρύθμισης και με αυτόν τον τρόπο εξοικονομεί χρόνο από την ρύθμιση της υποδομής και επιτρέπει άμεση εστίαση στην υλοποίηση της εφαρμογής γνωρίζοντας ότι υπάρχει μια ισχυρή υποδομή που έχει δημιουργηθεί από το JHipster με την επιλογή της επιπρόσθετων ρυθμίσεων για την βελτιστοποίηση της λειτουργίας της εφαρμογής.

Εκτός από τις ρυθμίσεις επιτρέπει τη δήλωση του Μοντέλου της εφαρμογής στην JDL DSL του, και μπορεί να δημιουργήσει τα ανάλογα Entities, Services Και Controllers και στο Frontend και στο Backend.

Η πτυχιακή αυτή αρχικά δημιουργήθηκε με το JHipster χρησιμοποιώντας το Spring Framework για το backend και την React Library για το Frontend. Επιπρόσθετα προσέθεσε ρυθμίσεις και για αρκετά άλλες τεχνολογίες που χρησιμοποιήθηκαν.

