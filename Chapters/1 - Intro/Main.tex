\chapter{Εισαγωγή}
Η βιομηχανία της έβδομης τέχνης έχει αναπτυχθεί ραγδαία. Η τεχνολογική επανάσταση της εποχής μας, έχει οδηγήσει στη ριζική μεταβολή της παραγωγής και αναπαραγωγής ταινιών, καθώς και στην  ίδρυση νέων εταιριών παραγωγής, στην αυξανόμενη δημιουργία ταινιών και νέων τεχνολογιών για την υποστήριξή τους.

Στη βιομηχανία του κινηματογράφου, όπως και σε κάθε άλλη, η βιωσιμότητα των εταιριών βασίζεται σε καίριους δείκτες απόδοσης (KPIs). Οι δείκτες αυτοί παρακολουθούνται και χρησιμοποιούνται από τους εργαζομένους στο χώρο του κινηματογράφου σε διάφορες θέσεις. 

Για παράδειγμα, ένας αναλυτής πρέπει να είναι συνεχώς ενήμερος για τις εξελίξεις έτσι ώστε να είναι σε θέση να κρίνει την απόδοση των ταινιών, των σχετικών συνεργασιών με ορισμένες χώρες ή εταιρίες, των ηθοποιών. Πρέπει να μπορεί να προβλέψει τις επερχόμενες τάσεις στις προτιμήσεις του κοινού και να παρουσιάσει καίρια στοιχεία για τον ανταγωνισμό. Ένας κυνηγός ταλέντων πρέπει να παρακολουθεί συνεχώς την πορεία νέων ηθοποιών, παραγωγών ταινιών , σκηνοθετών, οπερατέρ και γενικότερα το σύνολο των ρόλων σχετιζόμενων με μια ταινία. Πρέπει επίσης να μπορεί να προβλέψει την αλληλεπίδραση που έχουν οι συντελεστές αυτοί μεταξύ τους. Πόσο καλά θα συνεργάζονται οι ηθοποιοί μεταξύ τους, πόσο καλή συνεργασία θα έχουν οι ηθοποιοί με το συνεργείο παραγωγής και ειδικότερα με τους παραγωγούς, με τους σκηνοθέτες ακόμα και με τους συγγραφείς. Να μπορεί να βρει το σωστό άτομο για την εκάστοτε θέση. 

Αυτές οι εργασίες παλιότερα γινόταν πολύ πιο εύκολα καθώς δεν υπήρχε τόσος μεγάλος κορεσμός στην βιομηχανία. Τα τελευταία χρόνια με την ραγδαία αύξηση της ζήτησης αλλά και της παραγωγής, τα διαθέσιμα δεδομένα και όγκος τους, έχουν εκτοξευθεί, δημιουργώντας την ανάγκη για εργαλεία τα οποία να μπορούν να συλλέγουν αυτόν τον όγκο δεδομένων και να τα εμφανίζουν με έναν κατανοητό τρόπο που να διευκολύνει την ανάλυση και την ανάδειξη ταλέντων. Πλέον είναι πολύ πιο δύσκολο να μπορέσει κάποιος να συλλέξει τόσα δεδομένα και να βγάλει ένα σαφές συμπέρασμα. 

Όσο όμως εξελίχθηκε η βιομηχανία του κινηματογράφου, εξελίχθηκε και η τεχνολογία τα τελευταία χρόνια, παρέχοντας τα απαραίτητα εργαλεία που καθιστούν εφικτή τη διαχείριση αυτών των δεδομένων. Η εν λόγω πτυχιακή, έχει ως σκοπό να παρουσιάσει αυτά τα εργαλεία μέσω μιας πρακτικής απεικόνισης των δεδομένων, καλύπτοντας την ανάγκη για τη συλλογή, κατηγοριοποίηση και επεξεργασία ενός μεγάλου όγκου δεδομένων, και εν τέλει την παρουσίασή τους με έναν κατανοητό τρόπο για την εύκολη ανάλυση τους από τους ειδικούς. 

Πιο συγκεκριμένα, η πτυχιακή έχει δύο (2) κομμάτια; τον Server και τον Client. O Server συλλέγει δεδομένα από διάφορες υπηρεσίες ανοικτών δεδομένων βιομηχανίας κινηματογράφου (OpenData), συσχετίζει τα δεδομένα των διάφορων υπηρεσιών μεταξύ τους, και στη συνέχεια δημιουργεί έναν πίνακα σχέσεων μεταξύ όλων των δεδομένων για περαιτέρω κατηγοριοποίηση. Αφού γίνει ο συσχετισμός των δεδομένων, τα κατηγοριοποιεί, αλλάζει τη μορφή τους και τα αποθηκεύει σε μια βάση δεδομένων. Ο Client είναι αυτός που τα παίρνει από την βάση δεδομένων και τα εμφανίζει με έναν κατανοητό τρόπο στον τελικό χρήστη.

\section{Δομή της πτυχιακής}
\begin{itemize}
    \item Στο κεφάλαιο δύο περιγράφουμε τα OpenData, τι είναι και πως χρησιμοποιούντε
    \item Στο κεφάλαιο τρια παρουσιάζουμε τις τεχνολογίες που χρησιμοποιήθηκαν για την ανάπτυξη της πτυχιακής
    \item Στο κεφάλαιο τέσσερα ?!Stories?!
    \item Στο κεφάλαιο πέντε περιγράφουμε την αρχιτεκτονική της πτυχιακής
    \item Στο κεφάλαιο έξι είναι το εγχειρίδιο χρήσης.
    \item Στο κεφάλαιο επτά υπάρχουν τα συμπεράσματα
\end{itemize}
\section{Τεχνολογίες που χρησιμοποιήθηκαν}
\begin{itemize}
    \item \textbf{Server}
    \begin{enumerate}
        \item Java
        \item Parallel Programming
        \item Spring Framework
        \item Liquibase
    \end{enumerate}
    \item \textbf{Client}
    \begin{enumerate}
        \item TypeScript
        \item React Library
        \item Redux Framework
        \item CoreUI Framework
        \item HighCharts Library
    \end{enumerate}
    \item \textbf{Εξωτερικές Υπηρεσίες}
    \begin{enumerate}
        \item PostgreSQL
        \item ElasticSearch
        % \item Redis
    \end{enumerate}
    \item \textbf{Εργαλεία για την ανάπτυξη}
    %..+
    \begin{enumerate}
        \item Εργαλείο ανάπτυξης κώδικα Intellij Idea της Jetbrains
        \item Εργαλείο διαχείρισης βάσεων δεδομένων DataGrip της Jetbrains
        \item Εργαλείο διαχείρισης της ElasticSearch, Kibana
        \item Docker
        \item Εργαλείο διαχείρισης εκδόσεων κώδικα GitLab
        \item Εργαλείο CI, GitLab-CI
        \item JHipster
    \end{enumerate}
\end{itemize}