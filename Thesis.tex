\documentclass[oneside, 12pt]{book}
\usepackage{icdthesisUTF}
\usepackage{tabularx} 
\usepackage{epsfig}
%Τα παρακάτω είναι υποχρεωτικά:

\renewcommand{\thesistitle}{Οπτικοποίηση  δεδομένων βιομηχανίας κινηματογράφου με χρήση API δημόσιων ανοικτών δεδομένων}
\renewcommand{\thesisauthor}{Νικόλαος Μαυρόπουλος ( 3783 )}
\renewcommand{\thesisauthorabbrv}{Νικόλαος Μαυρόπουλος}
\renewcommand{\thesisauthorinitials}{NM}
\renewcommand{\thesissupervisor}{Ν. Πεταλίδης}
\renewcommand{\thesismonth}{Σεπτέμβριος}
\renewcommand{\thesisyear}{2020}

% Η βιβλιογραφία
\addbibresource{Thesis.bib}

\begin{document}
% Υποχρεωτικά τα παρακάτω:
\Titlepage
\Declarationpage
\begin{Abstract}
Στην παρούσα πτυχιακή θα εξετάσουμε πως με την χρήστη των Open Data μπορούν να δημιουργηθούν νέες ιδέες και τεχνολογίες με σκοπό της εξέλιξη της κοινωνίας και της τεχνολογίας. 
Σε αυτήν την εργασία παρουσιάζεται ένα εργαλείο που με βάση την χρήση των ανοιχτών δεδομένων (Open Data) παρουσιάζει συνδυαστικά και αναλυτικά δεδομένα, από 2 υπηρεσίες, της βιομηχανίας του κινηματογράφου με σκοπό την ευκολότερη επιτέλεση των εργασιών ατόμων που δουλεύουν στην βιομηχανία του κινηματογράφου αλλά και την ευκολότερη κατανόηση αυτών των δεδομένων. Τα δεδομένα παρουσιάζονται σε ένα φιλικό προς τον χρήστη γραφικό περιβάλλον με την χρήστη γραφημάτων, χαρτών κ.α
Η εφαρμογή χωρίζεται σε 2 μέρη, τον Client και τον Server και χρησιμοποιούνται διάσημες τεχνολογίες για την επίτευξη των στόχων της πτυχιακής.
\end{Abstract}
\tableofcontents

%Μόνο εφόσον θέλετε χωριστό πίνακα για εικόνες και πίνακες
\listoftables
\listoffigures

%Προαιρετικά
% \begin{Preface}
% Εδώ μπορεί να μπει πρόλογος. (Δεν είναι απαραίτητο).
% \end{Preface}

%Προαιρετικά
% \begin{Acknowledgement}
% Ευχαριστίες (στο μπαμπά, στη μαμά, κτλ)
% \end{Acknowledgement}
%Προαιρετικά
% \begin{Definitions}
% Ορισμοί εννοιών που μπορεί να είναι χρήσιμοι. Για παράδειγμα:

% \begin{description}
% \item [\LaTeX] Σύστημα στοιχειοθεσίας κειμένων
% \end{description}

% \end{Definitions}

%Από εδώ αρχίζει το κείμενό σας

\chapter{Εισαγωγή}
Η βιομηχανία της έβδομης τέχνης έχει αναπτυχθεί ραγδαία. Η τεχνολογική επανάσταση της εποχής μας, έχει οδηγήσει στη ριζική μεταβολή της παραγωγής και αναπαραγωγής ταινιών, καθώς και στην  ίδρυση νέων εταιριών παραγωγής, στην αυξανόμενη δημιουργία ταινιών και νέων τεχνολογιών για την υποστήριξή τους.

Στη βιομηχανία του κινηματογράφου, όπως και σε κάθε άλλη, η βιωσιμότητα των εταιριών βασίζεται σε καίριους δείκτες απόδοσης (KPIs). Οι δείκτες αυτοί παρακολουθούνται και χρησιμοποιούνται από τους εργαζομένους στο χώρο του κινηματογράφου σε διάφορες θέσεις. 

Για παράδειγμα, ένας αναλυτής πρέπει να είναι συνεχώς ενήμερος για τις εξελίξεις έτσι ώστε να είναι σε θέση να κρίνει την απόδοση των ταινιών, των σχετικών συνεργασιών με ορισμένες χώρες ή εταιρίες, των ηθοποιών. Πρέπει να μπορεί να προβλέψει τις επερχόμενες τάσεις στις προτιμήσεις του κοινού και να παρουσιάσει καίρια στοιχεία για τον ανταγωνισμό. Ένας κυνηγός ταλέντων πρέπει να παρακολουθεί συνεχώς την πορεία νέων ηθοποιών, παραγωγών ταινιών , σκηνοθετών, οπερατέρ και γενικότερα το σύνολο των ρόλων σχετιζόμενων με μια ταινία. Πρέπει επίσης να μπορεί να προβλέψει την αλληλεπίδραση που έχουν οι συντελεστές αυτοί μεταξύ τους. Πόσο καλά θα συνεργάζονται οι ηθοποιοί μεταξύ τους, πόσο καλή συνεργασία θα έχουν οι ηθοποιοί με το συνεργείο παραγωγής και ειδικότερα με τους παραγωγούς, με τους σκηνοθέτες ακόμα και με τους συγγραφείς. Να μπορεί να βρει το σωστό άτομο για την εκάστοτε θέση. 

Αυτές οι εργασίες παλιότερα γινόταν πολύ πιο εύκολα καθώς δεν υπήρχε τόσος μεγάλος κορεσμός στην βιομηχανία. Τα τελευταία χρόνια με την ραγδαία αύξηση της ζήτησης αλλά και της παραγωγής, τα διαθέσιμα δεδομένα και όγκος τους, έχουν εκτοξευθεί, δημιουργώντας την ανάγκη για εργαλεία τα οποία να μπορούν να συλλέγουν αυτόν τον όγκο δεδομένων και να τα εμφανίζουν με έναν κατανοητό τρόπο που να διευκολύνει την ανάλυση και την ανάδειξη ταλέντων. Πλέον είναι πολύ πιο δύσκολο να μπορέσει κάποιος να συλλέξει τόσα δεδομένα και να βγάλει ένα σαφές συμπέρασμα. 

Όσο όμως εξελίχθηκε η βιομηχανία του κινηματογράφου, εξελίχθηκε και η τεχνολογία τα τελευταία χρόνια, παρέχοντας τα απαραίτητα εργαλεία που καθιστούν εφικτή τη διαχείριση αυτών των δεδομένων. Η εν λόγω πτυχιακή, έχει ως σκοπό να παρουσιάσει αυτά τα εργαλεία μέσω μιας πρακτικής απεικόνισης των δεδομένων, καλύπτοντας την ανάγκη για τη συλλογή, κατηγοριοποίηση και επεξεργασία ενός μεγάλου όγκου δεδομένων, και εν τέλει την παρουσίασή τους με έναν κατανοητό τρόπο για την εύκολη ανάλυση τους από τους ειδικούς. 

Πιο συγκεκριμένα, η πτυχιακή έχει δύο (2) κομμάτια; τον Server και τον Client. O Server συλλέγει δεδομένα από διάφορες υπηρεσίες ανοικτών δεδομένων βιομηχανίας κινηματογράφου (OpenData), συσχετίζει τα δεδομένα των διάφορων υπηρεσιών μεταξύ τους, και στη συνέχεια δημιουργεί έναν πίνακα σχέσεων μεταξύ όλων των δεδομένων για περαιτέρω κατηγοριοποίηση. Αφού γίνει ο συσχετισμός των δεδομένων, τα κατηγοριοποιεί, αλλάζει τη μορφή τους και τα αποθηκεύει σε μια βάση δεδομένων. Ο Client είναι αυτός που τα παίρνει από την βάση δεδομένων και τα εμφανίζει με έναν κατανοητό τρόπο στον τελικό χρήστη.

\section{Δομή της πτυχιακής}
\begin{itemize}
    \item Στο κεφάλαιο δύο περιγράφουμε τα OpenData, τι είναι και πως χρησιμοποιούντε
    \item Στο κεφάλαιο τρια παρουσιάζουμε τις τεχνολογίες που χρησιμοποιήθηκαν για την ανάπτυξη της πτυχιακής
    \item Στο κεφάλαιο τέσσερα ?!Stories?!
    \item Στο κεφάλαιο πέντε περιγράφουμε την αρχιτεκτονική της πτυχιακής
    \item Στο κεφάλαιο έξι είναι το εγχειρίδιο χρήσης.
    \item Στο κεφάλαιο επτά υπάρχουν τα συμπεράσματα
\end{itemize}
\section{Τεχνολογίες που χρησιμοποιήθηκαν}
\begin{itemize}
    \item \textbf{Server}
    \begin{enumerate}
        \item Java
        \item Parallel Programming
        \item Spring Framework
        \item Liquibase
    \end{enumerate}
    \item \textbf{Client}
    \begin{enumerate}
        \item TypeScript
        \item React Library
        \item Redux Framework
        \item CoreUI Framework
        \item HighCharts Library
    \end{enumerate}
    \item \textbf{Εξωτερικές Υπηρεσίες}
    \begin{enumerate}
        \item PostgreSQL
        \item ElasticSearch
        % \item Redis
    \end{enumerate}
    \item \textbf{Εργαλεία για την ανάπτυξη}
    %..+
    \begin{enumerate}
        \item Εργαλείο ανάπτυξης κώδικα Intellij Idea της Jetbrains
        \item Εργαλείο διαχείρισης βάσεων δεδομένων DataGrip της Jetbrains
        \item Εργαλείο διαχείρισης της ElasticSearch, Kibana
        \item Docker
        \item Εργαλείο διαχείρισης εκδόσεων κώδικα GitLab
        \item Εργαλείο CI, GitLab-CI
        \item JHipster
    \end{enumerate}
\end{itemize}
\chapter{OpenData}

Open Data είναι η ιδέα ότι ορισμένα δεδομένα πρέπει να είναι διαθέσιμα δωρεάν σε όλον τον κόσμο για να τα χρησιμοποιεί ή να τα αναπαράγει όπως επιθυμεί χωρίς περιορισμούς από νόμους πνευματικών δικαιωμάτων, πατέντες η μηχανισμούς ελέγχου. Οι στόχοι του κινήματος των open-source data, είναι κοινός με των άλλων κινημάτων open-source όπως open-source software, hardware κ.ο.κ. \citep{wiki:opendata}.
Παραδόξως η ανάπτυξη του κινήματος open-source data είναι παράλληλη με την αύξηση των δικαιωμάτων πνευματικής ιδιοκτησίας.

Αν κάποιος αναρωτιέται γιατί είναι τόσο σημαντικό να είναι σαφές τι σημαίνει Ανοιχτά Δεδομένα και σε τι είναι χρήσιμος αυτός ο ορισμός, υπάρχει μια απλή απάντηση: η διαλειτουργικότητα.

Η διαλειτουργικότητα δηλώνει τη δυνατότητα διαφορετικών συστημάτων να λειτουργούν μαζί (διαλειτουργούν). Σε αυτή τη συγκεκριμένη περίπτωση, γίνεται αναφορά στη δυνατότητα να διαλειτουργούν –ή να αναμιγνύουν- διαφορετικά σύνολα δεδομένων.

Η διαλειτουργικότητα (interoperability) είναι σημαντική επειδή επιτρέπει στις διαφορετικές συνιστώσες να λειτουργούν μαζί. Αυτή η δυνατότητα διαμοίρασης και σύνδεσης συνιστωσών έχει θεμελιώδη σημασία για τη δόμηση μεγαλύτερων και πιο πολύπλοκων συστημάτων. Χωρίς τη δυνατότητα διαλειτουργικότητας αυτό γίνεται σχεδόν αδύνατο – απόδειξη η διάσημη ιστορία του Πύργου της Βαβέλ, όπου η αδυναμία επικοινωνίας (διαλειτουργίας) οδήγησε στην ολοκληρωτική κατάρρευση της προσπάθειας οικοδόμησης του. \citep{github:opendata}

Δυστυχώς το τοπίο των OpenData είναι πολύ θολό καθώς οι εταιρίες που προσφέρουν αυτά τα δεδομένα, πέρα από τους κυβερνητικούς οργανισμούς δεν ακολουθούν κάποια στάνταρ, και υπάρχουν διαφορετικοί περιορισμοί στην χρήση και αναπαραγωγή αυτών των δεδομένων. 

Είναι σημαντικό να υπάρξει ένα στάνταρ η μια αρμόδια αρχή που να υπαγορεύει ακριβώς πως τα OpenData θα χρησιμοποιούνται γενικότερα αλλά επίσης είναι σημαντικό όλο και περισσότερες εταιρίες να υιοθετήσουν αυτό το μοντέλο, καθώς με το "άνοιγμα" των δεδομένων δημιουργούνται νέες ιδέες για νέες τεχνολογίες και αναπτύσσεται γενικότερα η τεχνολογία και η κοινωνία.

Τα open source data (OpenData) δεν είναι πολύ διαφορετικά από το open source sofwtare (ή όπως χρησιμοποιείται ευρέως) OpenSource. Το κίνημα του open source software είναι ένα κίνημα για δωρεάν και ελεύθερο λογισμικό που η κοινωνία μπορεί να συμμετέχει στην εξέλιξη του και στην διασφάλιση της ποιότητας του, και το κίνημα των open source data είναι ένα κίνημα για δωρεάν και ελεύθερα δεδομένα τα οποία μπορεί να αξιοποιήσει η κοινωνία για την δημιουργία open source software. 

Έχοντας παραπάνω πηγές open source data, θα δημιουργούνται παραπάνω ιδέες και παραπάνω εργαλεία που θα συμβάλλουν σημαντικά στην εξέλιξη μας. Όσες παραπάνω ιδέες και παραπάνω εργαλεία τόσα παραπάνω open source software θα γεννηθούν για την αξιοποίηση αυτών των δεδομένων. Η παρούσα πτυχιακή είναι ένα project το οποίο γεννήθηκε εξαιτίας των open source data εξαρχής. Δεν θα ήταν εφικτή η υλοποίηση της χωρίς αυτά. 

~\citep{Wikipedia_BibTeX}
% \chapter{Εισαγωγή}
% \leftmark\rightmark
% \section{Η τυπογραφία σήμερα}
% Αυτή είναι η αναφορά σε ένα άρθρο περιοδικού:\citep{Schmidt98}.Αυτή
% είναι η αναφορά σε ένα βιβλίο:\citep{goosens93}. Αυτή είναι η αναφορά
% σε ένα ελληνικό βιβλίο:\citep{Chatzigeorgiou05}. Βιβλίο στα ελληνικά
% με ξένο συγγραφέα:\citep{Sommerville09}. Άρθρο σε
% συνέδριο~\citep{4343930}. 

% Τέλος αναφορά σε ιστοσελίδα:~\citep{Wikipedia_BibTeX}.

% Εδώ αναφερόμαστε στο σχήμα~\ref{fig:image1}:
% \begin{figure}[h]
%   \centering
%   \includegraphics[width=35mm]{lion.png}
%   \caption{Παράδειγμα εικόνας}
%   \label{fig:image1}
% \end{figure}
% dsasdadadasdasdas
% ad
% asd
% asd

% και εδώ στον πίνακα~\ref{tab:table1}:
% \begin{table}[h]
%   \centering
%   \caption{Παράδειγμα πίνακα}
%  \begin{tabularx}{\linewidth}[h]{|XXX|}%
% \hline
% \hline
% Κίνητρα & Παραδείγματα ευρημάτων & Αριθμός μελετών\\
% \hline
% Ταύτιση με το έργο & Ξεκάθαροι στόχοι &20\\
% Καλό management & Ομαδικότητα &16\\
% Συμμετοχή υπαλλήλων & Συμμετοχή στις αποφάσεις&16\\
% Προοπτικές εξέλιξης & Προοπτικές προαγωγής&15\\
% Ποικιλία στην εργασία & Καλή χρήση ικανοτήτων& 14\\
% Αίσθηση του να ανήκεις κάπου& Υποστηρικτικές σχέσεις&14\\
% Αμοιβές και κίνητρα & Αυξημένος μισθός& 14\\
% \hline
% \hline
% \end{tabularx}
%   \label{tab:table1}
% \end{table}
% \appendix
% \chapter{Συνοπτικός οδηγός χρήσης \LaTeX}
% Εδώ βάζετε ότι θα έμπαινε σε παράρτημα.
% \texttt{Δοκιμή σε mono-space}
%Προαιρετικά

\begin{Glossary}
Το γλωσσάρι μπορεί να είναι χρήσιμο αν χρησιμοποιείτε πολλά ακρώνυμα
και συντομογραφίες. Για παράδειγμα
\begin{description}
\item[TCP]Transmission Control Protocol
\end{description}
\end{Glossary}
\printbibliography
\lastpageinfo
\end{document}
